%
%%%%%%%%%%%%%%%%%%%%%%%%%%%%%%%%%%%%%%%%
\pagenumbering{roman}                   %serve per mettere i numeri romani
\chapter*{Introduzione}                 %crea l'introduzione (un capitolo
                                        %   non numerato)
%%%%%%%%%%%%%%%%%%%%%%%%%%%%%%%%%%%%%%%%%imposta l'intestazione di pagina
\rhead[\fancyplain{}{\bfseries
INTRODUZIONE}]{\fancyplain{}{\bfseries\thepage}}
\lhead[\fancyplain{}{\bfseries\thepage}]{\fancyplain{}{\bfseries
INTRODUZIONE}}
%%%%%%%%%%%%%%%%%%%%%%%%%%%%%%%%%%%%%%%%%aggiunge la voce Introduzione
                                        %   nell'indice
\addcontentsline{toc}{chapter}{Introduzione}

I progressi nelle tecnologie dell'informazione e della comunicazione, e in particolare nell'ingegneria multimediale, di rete e del software, hanno promosso una nuova generazione di ambienti di apprendimento online.

Molte organizzazioni, sia pubbliche che private, sfruttano le nuove tecnologie per offrire prodotti e servizi didattici e formativi a tutti i livelli. In particolare, l'ampia disponibilità di risorse educative è un obiettivo comune per università, biblioteche, archivi e altre istituzioni ad alta intensità di conoscenza.

\vspace{5mm}

Il progetto di tesi vede come oggetto lo sviluppo di un'applicazione web che permetta a studenti universitari di sottoporre le proprie generalità a un sistema di raccomandazione di corsi accademici e relative risorse didattiche associate, dando rilevanza anche alle eventuali disabilità possedute dallo studente.

Il modello di raccomandazione si basa su regole che estendono e inferiscono nuove relazioni semantiche tra i componenti dell'ontologia \textit{Pathadora}, progettata ad-hoc dal collega Sokol Guri per rappresentare i componenti del dominio, incorporando ontologie già esistenti che modellano l'accessibilità delle risorse e l’organizzazione strutturale delle istituzioni educative. 

\vspace{5mm}

L'applicazione permetterà inoltre ai docenti universitari di associare nuove risorse ai corsi di cui sono titolari, su cui verranno generati automaticamente metadati relativi alle proprietà del file e al suo contenuto. Questi metadati verranno utilizzati dal sistema di raccomandazione per suggerire risorse il cui grado di accessibilità secondo determinati aspetti, come la modalità di fruizione del contenuto o la sua trasformabilità, è compatibile con le eventuali disabilità possedute dallo studente.

\vspace{5mm}

Verranno inoltre analizzate varie tecniche di generazione automatica di metadati, il cui scopo è facilitare l'estrazione di informazioni da associare alle risorse.

\vspace{5mm}

La tesi è strutturata in quattro capitoli:
\begin{enumerate}
\item \textbf{Stato dell'arte}, verranno descritte le tematiche principali che interessano i temi dell'estrazione automatica di metadati e l'utilizzo di questi ultimi nel contesto dell'e-learning;
\item \textbf{Contesto del progetto}, viene descritto in cosa consiste il progetto Pathadora e le relative funzionalità;
\item \textbf{Tecnologie}, analizza i linguaggi e le principali tecnologie utilizzate all’interno del progetto;
\item \textbf{Implementazione}, descrive le tecniche di implementazione impiegate nelle diverse parti dell’applicazione.
\end{enumerate}

%%%%%%%%%%%%%%%%%%%%%%%%%%%%%%%%%%%%%%%%%non numera l'ultima pagina sinistra
\clearpage{\pagestyle{empty}\cleardoublepage}