%%%%%%%%%%%%%%%%%%%%%%%%%%%%%%%%%%%%%%%%%non numera l'ultima pagina sinistra
\clearpage{\pagestyle{empty}\cleardoublepage}
%%%%%%%%%%%%%%%%%%%%%%%%%%%%%%%%%%%%%%%%%per fare le conclusioni
\chapter*{Conclusioni}
%%%%%%%%%%%%%%%%%%%%%%%%%%%%%%%%%%%%%%%%%imposta l'intestazione di pagina
\rhead[\fancyplain{}{\bfseries
CONCLUSIONI}]{\fancyplain{}{\bfseries\thepage}}
\lhead[\fancyplain{}{\bfseries\thepage}]{\fancyplain{}{\bfseries
CONCLUSIONI}}
%%%%%%%%%%%%%%%%%%%%%%%%%%%%%%%%%%%%%%%%%aggiunge la voce Conclusioni
                                        %   nell'indice
\addcontentsline{toc}{chapter}{Conclusioni}

L'obiettivo di questa tesi è quello di illustrare le tematiche e le fasi di lavoro che hanno interessato lo sviluppo di un sistema di raccomandazione di facoltà, corsi e risorse didattiche e della generazione di metadati che descrivano il contenuto delle risorse didattiche prese a carico.

\vspace{5mm}

Lo sviluppo dell'interfaccia web e delle tecniche di generazione di metadati è stato effettuato in parallelo con lo sviluppo del modello di raccomandazione del percorso accademico e del percorso di apprendimento dello studente, che si basa su regole
che estendono e inferiscono nuove relazioni semantiche tra i componenti dell’ontologia che modella il dominio.

\vspace{5mm}

L'applicazione sviluppata è stata testata su più browser (Google Chrome, Mozilla Firefox, Microsoft Edge), risulta funzionante e restituisce risultati corretti se configurata adeguatamente.

Lo sviluppo di tecniche di generazione automatica di metadati si è focalizzato sulle risorse didattiche di tipo testuale; un possibile approfondimento potrebbe comprendere la generazione di metadati relativi a risorse multimediali, basate sull'estrazione del testo udibile nella sorgente audio come discusso nella capitolo dello stato dell'arte.
