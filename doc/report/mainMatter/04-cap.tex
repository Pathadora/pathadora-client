\clearpage{\pagestyle{empty}\cleardoublepage}
\chapter{Implementazione}                %crea il capitolo
%%%%%%%%%%%%%%%%%%%%%%%%%%%%%%%%%%%%%%%%%imposta l'intestazione di pagina
\section{Frontend}
Il frontend è stato sviluppato utilizzando Angular 12, utilizzando Angular CLI per installare librerie ed eseguire l'applicazione. La progettazione dei componenti è stata preceduta dalla realizzazione dei mockup.

L'applicazione è suddivisa in sottocartelle per ciascun componente:
\begin{itemize}
	\item \texttt{home}: componente il cui contenuto è accessibile solo a seguito dell'autenticazione. Nel caso l'utente autenticato sia uno studente, verrà renderizzato il componente \texttt{dashboard-user};
	\item \texttt{login}: permette l'accesso dell'utente registrato nel sistema, sarà accessibile a chi non ha eseguito l'accesso;
	\item \texttt{register}: permette la registrazione di un utente. È utilizzato per due route diverse, relative alla registrazione dello studente e alla registrazione del docente, per cui verranno richiesti campi diversi;
	\item \texttt{dashboard-user}: componente attraverso il quale vengono sottomesse al sistema di raccomandazione le diverse richieste per ottenere il learning path utilizzando come dati di input le informazioni associate all'utente autenticato. L'invio delle richieste avviene attraverso tre form corrispondenti all'invio di informazioni per ciascun step della produzione del learning path:
\begin{itemize}
\item \textit{richiesta dei dipartimenti e delle facoltà consigliate}: vengono utilizzate come input le informazioni dello studente associate al momento della registrazione;
\item \textit{richiesta dei corsi consigliati}: una volta ottenuti i dipartimenti e le facoltà consigliate, l'utente può inviare la richiesta per visualizzare i corsi consigliati dal sistema di raccomandazione selezionando il dipartimento, la facoltà e l'anno di interesse;
\item \textit{richiesta delle risorse consigliate}: una volta selezionato il corso di interesse tra quelli consigliati, verranno visualizzate le risorse suggerite sulla base dei dati dello studente.
\end{itemize}
	\item \texttt{profile-teacher}: pannello che permette al docente di associare nuovi corsi al suo profilo tra quelli memorizzati nel sistema;
	\item \texttt{dashboard-courses}: vengono visualizzate le informazioni dei corsi. Gli amministratori possono modificarle;
	\item \texttt{dashboard-resources}: pannello visualizzabile in home dai docenti e dagli amministratori, permette il caricamento di nuove risorse didattiche ai corsi di cui sono titolari.
\end{itemize}

La gestione delle richieste al server backend è contenuta nei metodi di servizi dedicati.

\subsection{Servizio di autenticazione}
Il servizio di autenticazione contiene la logica delle funzioni di login, logout e registrazione, utilizzabili dai componenti del sistema.

I subject e gli observable della libreria RxJS sono usati per tenere traccia dell'utente attualmente loggato e notificare del login e del logout, mediante il metodo \texttt{this.currentUserSubject.next()}, i componenti che eseguono \texttt{subscribe()} sull'oggetto \texttt{currentUser}.

Il metodo \texttt{login()} invia al server una richiesta POST per l'autenticazione utilizzando le credenziali dell'utente. Se l'utente è presente nel database, le sue informazioni verranno memorizzate nel local storage per tenere traccia del login in tutte le pagine.

Il metodo \texttt{login()} rimuove dal local storage le informazioni relative all'utente loggato e imposta \texttt{currentUserSubject} a \texttt{null}, notificando i subscriber del logout.

Il metodo \texttt{register()} invia al server una richiesta POST per la registrazione dell'utente, per poi proseguire con la chiamata di \texttt{login()}.

\subsection{Classi di utility}
L'\textbf{Auth Guard} è un Route Guard usato per limitare l'accesso a determinati route a utenti autorizzati, passato come parametro \texttt{canActivate} ai percorsi di route interessati. Il metodo \texttt{canActivate()}, invocato ogni volta che qualcuno tenterà di accedere al route associato, restituisce un valore booleano che indica se consentire o meno la navigazione. Se l’utente non è autenticato, verrà reindirizzato alla route di login.

\vspace{5mm}

L'\textbf{Error Interceptor} intercetta le risposte delle richieste API e provvede a eseguire il logout dall'applicazione in caso di errori. Implementa la classe HttpInterceptor della libreria \texttt{@angular/common} ed è aggiunto tra i provider dell'applicazione. Grazie alle informazioni presenti nell'oggetto Provider, il Root Injector sa come creare e recuperare un'istanza dell'ErrorInterceptor quando serve e fornirla a qualsiasi componente, direttiva o altro servizio che ne faccia richiesta.

\vspace{5mm}

Il \textbf{JWT Interceptor} intercetta le richieste invocate per aggiungere un token JWT di autenticazione nell'header qualora l'utente dovesse essere loggato.

\section{Backend}
Oltre alle richieste eseguite al Pathadora Engine per ottenere il learning path suggerito e aggiungere individual nell'ontologia, viene utilizzato un server Express per gestire l'autenticazione degli utenti e accedere alle proprietà degli oggetti aggiunti memorizzati in un database MongoDB.

Il server viene avviato lanciando il file \texttt{server.js}, attraverso il quale viene creata un'istanza Express, viene stabilita una connessione col database MongoDB e vengono definite le diverse route utilizzate dall'applicazione.

Nello sviluppo è stato utilizzato il tool Robo3T per manipolare gli oggetti nelle collezioni del database. La validità delle richieste è stata testata con Postman.

Le route sono suddivise nei seguenti moduli, ciascuno dei quali fa riferimento a uno schema Mongoose utilizzato nel database:
\begin{itemize}
	\item \texttt{users}
	\begin{itemize}
		\item \texttt{/register} (post): carica nel database un nuovo utente utilizzando le informazioni del corpo dell'oggetto richiesta e effettua il login di quest'ultimo assegnandogli un token. Viene inoltre eseguito l'inserimento di un individual Learner nell'ontologia Pathadora;
		\item \texttt{/} (get): restituisce i dati di tutti gli utenti registrati nel database;
		\item \texttt{/me} (get): restituisce i dati dell'utente loggato;
		\item \texttt{/courses} (get): restituisce i dati dei corsi associati all'utente autenticato, nel caso quest'ultimo abbia il ruolo \texttt{teacher};
		\item \texttt{/courses} (post): riceve come input un array di id di corsi e li associa al docente autenticato;
		\item \texttt{/courses/:id} (delete): rimuove il corso con id specificato dall'elenco dei corsi associati al docente; autenticato
	\end{itemize}
	\item \texttt{auth}
	\begin{itemize}
		\item \texttt{/} (get): restituisce le informazioni relative all'utente loggato (generalità, lingua, titolo posseduto, tipo di laurea futura, passioni, obiettivo, disabilità, corsi di cui è titolare nel caso abbia il ruolo \texttt{teacher});
		\item \texttt{/} (post): effettua il login dell'utente con email e password sottomesse generando un token.
	\end{itemize}
	\item \texttt{courses}
	\begin{itemize}
		\item \texttt{/} (get): restituisce i dati di tutti i corsi registrati nel database (nome, tipo di laurea, facoltà di appartenenza, lingua, semestre, numero di CFU, anno, tipo di corso, area scientifica, obbligatorietà, filepath delle risorse associate e relativi metadati);
		\item \texttt{/:id} (get): restituisce le informazioni associate al corso con id specificato;
		\item \texttt{/} (post): se l'utente autenticato ha il ruolo \texttt{admin}, registra un nuovo corso. Viene inoltre eseguito l'inserimento di un individual Course nell'ontologia Pathadora;
		\item \texttt{/:id} (post): se l'utente autenticato ha il ruolo \texttt{admin}, aggiorna il corso con id specificato;
		\item \texttt{/:id} (delete): se l'utente autenticato ha il ruolo \texttt{admin}, rimuove il corso con id specificato;
		\item \texttt{/resource/:course\_id} (post): se l'utente autenticato ha il ruolo \texttt{teacher}, associa una risorsa al corso con id associato. Il file viene caricato utilizzando il middleware Multer di Express, memorizzato in una directory di file statici accessibili mediante richiesta HTTP. Viene inoltre eseguito l'inserimento di un individual LearningObject nell'ontologia Pathadora.
	\end{itemize}
	\item \texttt{faculties}
	\begin{itemize}
		\item \texttt{/} (get): restituisce i dati di tutte le facoltà registrate nel database (nome, dipartimento di appartenenza);
		\item \texttt{/:id} (get): restituisce le informazioni associate alla facoltà con id specificato;
		\item \texttt{/} (post): se l'utente autenticato ha il ruolo \texttt{admin}, registra una nuova facoltà;
		\item \texttt{/:id} (post): se l'utente autenticato ha il ruolo \texttt{admin}, aggiorna la facoltà con id specificato;
		\item \texttt{/:id} (delete): se l'utente autenticato ha il ruolo \texttt{admin}, rimuove la facoltà con id specificato;
		\item \texttt{/department/:departmentid} (get): viene restituito l'elenco di facoltà appartenenti al dipartimento con l'id specificato.
	\end{itemize}
	\item \texttt{departments}
	\begin{itemize}
		\item \texttt{/} (get): restituisce i nomi di tutti i dipartimenti registrati nel database;
		\item \texttt{/:id} (get): restituisce le informazioni associate ai dipartimenti con id specificato;
		\item \texttt{/} (post): se l'utente autenticato ha il ruolo \texttt{admin}, registra un nuovo dipartimento;
		\item \texttt{/:id} (post): se l'utente autenticato ha il ruolo \texttt{admin}, aggiorna il dipartimento con id specificato;
		\item \texttt{/:id} (delete): se l'utente autenticato ha il ruolo \texttt{admin}, rimuove il dipartimento con id specificato.
	\end{itemize}
\end{itemize}

Per le route che richiedono la verifica dell'identità dell'utente loggato è utilizzata una funzione middleware che accede al token associato all'header della richiesta, decodificandolo con JWT per ottenere come output le informazioni dell'utente che potranno poi essere utilizzate dall'handler associato alla route.

\section{Estrazione e generazione dei metadati}
Al caricamento di una risorsa associata a un corso, viene invocato un processo Python che calcola una serie di metadati relativi alle proprietà della risorsa e del suo contenuto di cui il modello di profilazione farà uso nel calcolo del suggerimento delle risorse consigliate.

\subsection{Estrazione dei metadati intrinseci del file}
Al caricamento di una risorsa associata a un corso, i metadati a essa associati vengono estratti mediante la libreria \texttt{metadata-extract}, che a sua volta fa uso di diversi estrattori per più estensioni.

\subsection{Grado di leggibilità di Flesh-Kincaid}
Viene inoltre calcolato attraverso uno script Python un metadato relativo al grado di leggibilità del testo contenuto nel file, utilizzando la formula di \textbf{Flesh-Kincaid}. Più il valore è alto, più il testo risulta essere semplice da leggere.

\[ F=206,835-(84,6*S)-(1,015*P) \]

La difficoltà di lettura dipende da S e da P, rispettivamente il numero medio di sillabe contenute in una parola e il numero medio di parole contenute in una frase.

La scelta dei coefficienti è una conseguenza di un processo di affinamento del grado di istruzione di una persona in grado di leggere e comprendere la lingua inglese, ottimizzati in modo che venga restituito un valore compreso tra 0 e 100. Viene data più importanza al valore di S piuttosto che a P.\cite{readibility}

\subsection{Accessibilità delle immagini in un documento}
È stato prodotto uno script Python attraverso cui viene restituito un indicatore di accessibilità del testo contenuto nelle immagini presenti in un documento PDF.

Per ciascuna immagine presente nel documento, estrapolate mediante la libreria Python-tesseract, vengono catturate le aree che contengono testo ed estratto il colore dei caratteri in formato RGB (prelevando il colore più rilevante nella textbox analizzata). Viene poi calcolato il rapporto di contrasto tra questo colore e il colore di sfondo dell'immagine, seguendo i criteri di WCAG 2 dove il range di differenza di luminosità tra i due colori varia da 1 (ad esempio bianco su bianco) a 21 (ad esempio nero su bianco). 

Il valore minimo tra quelli calcolati rappresenta il valore dell'immagine meno accessibile e quindi più rappresentativa dell'accessibilità della risorsa.

Il livello minimo per cui il contenuto può essere considerato accessibile è 4.5.

\subsection{Dimensione minima del carattere in un documento}
È stato prodotto uno script Python attraverso cui viene restituito il valore della dimensione minima dei caratteri individuata in un documento PDF. Più è basso il valore meno il contenuto è accessibile agli utenti con problemi visivi.

Le informazioni relative al font dei caratteri vengono ricavate attraverso il parsing del contenuto del documento con le funzionalità ad alto livello fornite dalla libreria PDFMiner.

\subsection{Parole chiave}
Viene utilizzato l'algoritmo YAKE! per estrarre le 10 parole chiave più rilevanti contenute in un testo. A una keyword è associato un punteggio, più il valore è basso più la keyword è rilevante. La libreria permette la personalizzazione della lingua analizzata, della dimensione massima degli n-grammi risultanti, del numero di keyword più rilevanti restituite e della soglia di duplicazione delle parole, che rappresenta il grado per cui una parola può comparire in più keyword.

Il testo analizzato viene estratto se possibile dal documento attraverso la libreria Textract.

Come gli altri metadati questo risultato verrà aggiunto ai metadati della risorsa associata.